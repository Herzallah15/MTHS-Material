This exercise requires the openQCD software suite, available \href{https://luscher.web.cern.ch/luscher/openQCD/openQCD-2.4.2.tar.gz}{here}.

\begin{enumerate}
	\item Make sure the \texttt{Open MPI} and \texttt{OpenMP} libraries are installed.
	\item In the openQCD source code, edit the file \texttt{main/Makefile}.
	      Near the top, add a line \texttt{GCC=<>} to specify the C-compiler on your
	      system. Also specify \texttt{MPI\_HOME} to provide the location of the MPI libraries on your system, and \texttt{MPI\_INCLUDE} for the location of the mpi header files.
	\item In \texttt{include/global.h}, set \texttt{L0} to \texttt{L3} to simulate a $16^4$ lattice. The number of threads can be controlled by editing \texttt{L0\_TRD} to \texttt{L3\_TRD}. Return to the \texttt{main} directory, and execute the \texttt{make} command.
	\item In a directory to be used for running, create the \texttt{log}, \texttt{dat}, and \texttt{cnfg} directories, and download modified input file
	      \texttt{ym1.in}.
	\item Run a simulation using the \texttt{ym1} executable using the Wilson gauge action at $\beta=5.789$ approximately corresponding to a lattice spacing $a=0.14\,{\rm fm}$. Follow the plaquette to trace the thermalization process. Also, monitor the smoothed global topological charge. How many trajectories does it typically take for the charge to change sign?
	\item  Now change to $\beta=6.0$. To maintain an approximately constant physical volume, a $24^4$ lattice is required. Can you observe a `slowdown' in the global topological charge?
\end{enumerate}