
\newcommand{\ma}{\ensuremath{M}\xspace}
\newcommand{\mb}{\ensuremath{m_1}\xspace}
\newcommand{\mc}{\ensuremath{m_2}\xspace}
\newcommand{\eb}{\ensuremath{E_1}\xspace}
\newcommand{\ec}{\ensuremath{E_2}\xspace}

\newcommand{\pb}{\ensuremath{p_1}\xspace}
\newcommand{\pc}{\ensuremath{p_2}\xspace}
\newcommand{\pvecb}{\ensuremath{\textbf{p}_1}\xspace}
\newcommand{\pvecc}{\ensuremath{\textbf{p}_2}\xspace}
\newcommand{\pvecstar}{\ensuremath{\textbf{p}^{*}}\xspace}
\newcommand{\pmodvecb}{\ensuremath{\text{p}_1}\xspace}
\newcommand{\pmodvecc}{\ensuremath{\text{p}_2}\xspace}
\newcommand{\pmodvecstar}{\ensuremath{\text{p}^{*}}\xspace}

\be

\item Particle $A$ with mass \ma decays into two daughters with masses \mb
and \mc.  Derive the formula for the {\em break up momentum},
$\pmodvecstar = |\pvecb| = |\pvecc|$, in the rest frame of $A$.
Use your result to evaluate $\pmodvecstar$ for the decay \mbox{$\Delta(1232)\to p\pi$}.

\begin{solution}
    First calculate \eb:
    \begin{align*}
        P^2 = \ma^2 & = (\pb + \pc)^2 = \pb^2 + \pc^2 + 2\pb\cdot \pc                                                              \\
                    & = \mb^2 + \mc^2 + 2(\eb \ec - \pvecb\cdot \pvecc)                                                            \\
                    & = \mb^2 + \mc^2 + 2(\eb \ec + \pmodvecb^2) \qquad \rarrow \qquad \text{with }\pvecb = -\pvecc                \\
                    & = (2\mb^2 + 2\pmodvecb^2) - \mb^2 + \mc^2 + 2\eb(\ma-\eb) \qquad \rarrow \qquad \text{with } \ma = \eb + \ec \\
                    & \rarrow \eb = \frac{\ma^2 + \mb^2 - \mc^2}{2\ma}
    \end{align*}
    Now calculate $\pmodvecb$:
    \begin{align*}
        \pmodvecb^2 & = \eb^2 - \mb^2                                                                                                      \\
                    & = (\eb + \mb)(\eb - \mb)                                                                                             \\
                    & = \left(\frac{\ma^2 + \mb^2 - \mc^2 + 2\ma\mb}{2\ma}\right)\left(\frac{\ma^2 + \mb^2 - \mc^2 - 2\ma\mb}{2\ma}\right) \\
                    & \rarrow \pmodvecb = \frac{1}{2\ma}
        \left[ \left( (\ma+\mb)^2 - \mc^2 \right)
            \left( (\ma-\mb)^2 - \mc^2 \right)  \right ]^{\frac{1}{2}}
    \end{align*}
    I believe this is equivalent to the PDG formula:
    \begin{align*}
        \pmodvecb = \frac{1}{2\ma}
        \left[ \left( \ma^2 - (\mb+\mc)^2 \right)
            \left( \ma^2 - (\mb-\mc)^2 \right)  \right ]^{\frac{1}{2}}
    \end{align*}

    For the decay $\Delta(1232)\to p\pi$, $p^* \approx 230$~MeV/c

\end{solution}


\item Starting from the formula for the 2-body decay rate,
\begin{align*}
    \Gamma_{fi} = \frac{1}{2\ma} \int |\mathcal{M}_{fi}|^2
    \frac{\text{d}^3\pb}{(2\pi)^3 2\eb}
    \frac{\text{d}^3\pc}{(2\pi)^3 2\ec}\,
    (2\pi)^4 \delta^4(P - p_1 - p_2 )\,,
\end{align*}

% integrate out all variables
% except $\text{d}\Omega^* = \text{d}\cos\theta^*\text{d}\phi^*$ 

perform integrations using $\delta$ functions to obtain $\diff \Gamma/\diff\Omega$ in the centre of mass frame, $P = \begin{pmatrix}M\\\vec{0}\end{pmatrix}$,
to obtain the 2-body phase-space factor.

    {\em Hint: You may want to use the following property of the delta function:}
\begin{align*}
    \int_{-\infty}^{+\infty} g(x) \delta(f(x)) \text{d}x = \sum_{x_0} \frac{g(x_0)}{|\text{d}f/\text{d}x|_{x_0}}, %\int_{x_1}^{x_2} \delta(x-x_0)
    \text{ where } x_0 \in \{x:\, f(x)=0\}.
\end{align*}


\begin{solution}
    First integrate out \pvecc, imposing $\pvecc = -\pvecb$ via the delta function:
    \begin{align*}
        \Gamma_{fi} = \frac{1}{8\pi^2\ma} \int |\mathcal{M}_{fi}|^2
        \delta(\ma - \eb - \ec)
        \frac{\text{d}^3\pb}{4\eb\ec}
    \end{align*}
    Substitute $\text{d}^3\pb = \pmodvecb^2\text{d}\pmodvecb\text{d}\Omega$:
    \begin{align*}
        \Gamma_{fi} = \frac{1}{8\pi^2\ma} \int |\mathcal{M}_{fi}|^2
        \delta(\ma - \eb - \ec)
        \frac{\pmodvecb^2\text{d}\pmodvecb\text{d}\Omega}{4\eb\ec}
    \end{align*}
    Set:
    \begin{align*}
        f(\pmodvecb)                          & = M-\eb-\ec = M - \left(\mb^2+\pmodvecb^2\right)^\frac{1}{2}
        - \left(\mc^2+\pmodvecb^2\right)^\frac{1}{2}                                                         \\
        \frac{\partial f}{\partial \pmodvecb} & =
        -\pmodvecb\left[\left(\mb^2+\pmodvecb^2\right)^{-\frac{1}{2}}
        + \left(\mc^2+\pmodvecb^2\right)^{-\frac{1}{2}}\right]                                               \\
        g(\pmodvecb)                          & = \frac{\pmodvecb^2}{4\eb\ec}
        = \frac{\pmodvecb^2}{4\left(\mb^2+\pmodvecb^2\right)^\frac{1}{2}
        \left(\mc^2+\pmodvecb^2\right)^{-\frac{1}{2}}}                                                       \\
        \Gamma_{fi}                           & = \frac{1}{8\pi^2\ma} \int |\mathcal{M}_{fi}|^2
        \delta\left(f(\pmodvecb)\right) g(\pmodvecb) \text{d}\pmodvecb\text{d}\Omega                         \\
                                              & = \frac{1}{8\pi^2\ma}
        \left|\frac{\text{d}f}{\text{d}\pmodvecb}\right|_{\pmodvecstar}^{-1}
        \int |\mathcal{M}_{fi}|^2
        g(\pmodvecb)\delta(\pmodvecb - \pmodvecstar)\text{d}\pmodvecb\text{d}\Omega                          \\
                                              & = \frac{1}{8\pi^2\ma}
        \frac{\eb\ec}{\pmodvecstar(\eb+\ec)} \frac{(\pmodvecstar)^2}{4\eb\ec}
        \int |\mathcal{M}_{fi}|^2\text{d}\Omega                                                              \\
                                              & = \frac{\pmodvecstar}{32\pi^2\ma^2}
        \int |\mathcal{M}_{fi}|^2\text{d}\Omega
    \end{align*}
\end{solution}



% \item Use the phase-space factor to calculate the ratio of the decays $f_0(1500)\rightarrow \eta\eta$ and $f_0(1500)\rightarrow K\bar{K}$.
% 
% How does this compare to the measured values, and what can be concluded from this?
% 
%  \begin{solution}
%  \begin{align*}
%  \frac{\Gamma_{f_0(1500)\rightarrow \eta\eta}}{\Gamma_{f_0(1500)\rightarrow K\bar{K}}} 
%      = \frac{\pmodvecstar_{\eta\eta}}{\pmodvecstar_{K\bar{K}}} = \sqrt{\frac{M_{f_0(1500)}^2-4m_\eta^2}{M_{f_0(1500)}^2-4m_K^2}}
%      %\approx \frac{45.465\GeV}{45.500\GeV}
%      \approx 0.91
%  \end{align*}
% 
% This is roughly comparable to the measured ratio of 0.59, and could be interpreted as the $f_0(1500)$ having an exotic gluonic component which is ``flavour blind''. 
%  
%  \end{solution}

% \item Use the phase-space factor to calculate the ratio of the
% $Z\to\tau\tau$ to the $Z\to ee$ decay rates, assuming that the matrix element is a constant, which does not depend on the masses of the coupled leptons.
% 
% \begin{solution}
% \begin{align*}
% \frac{\Gamma_{Z\to\tau\tau}}{\Gamma_{Z\to ee}} 
%     = \frac{\pmodvecstar_{\tau\tau}}{\pmodvecstar_{ee}} = \sqrt{\frac{M_Z^2-4m_\tau^2}{M_Z^2-4m_e^2}}
%     \approx \frac{45.465\GeV}{45.500\GeV}
%     = 0.9992
% \end{align*}
% 
% \end{solution}

%\item Derive the 3-body phase-space element and use it to calculate the ratio
%of the decay rates for 
%\mbox{$K^{+}\to\pi^+\pi^+\pi^-$} and \mbox{$K^{+}\to\pi^+\pi^0\pi^0$}
%assuming the same matrix element for each.  
%
%{\bf Note from Will: I think we should drop this one. The Daliz covers enough, and 
%this is getting way to hard for them}

\ee
