Consider the (zero-momentum) pion interpolator

\begin{equation}
	O_{\pi^{+}}(t) = \sum_{\textbf{x}} \bar{d}(x) \, \gamma_5\, u(x). 
\end{equation}
This exercise concerns the computation of correlation functions using Wick's theorem, and the signal to noise problem. 
\begin{enumerate}
	\item Compute the correlation function $C(t) = \langle \bar{O}_{\pi^{+}}(t) \, O_{\pi^{+}}(0) \rangle$ in terms of the light quark propagator $D_{l}^{-1}(x-y)$. Use that result to conclude that (for asymptotically large separations)
		\begin{equation}
			|D_{l}^{-1}(x-y)| \sim {\rm e}^{-m_{\pi}|x-y|}.
		\end{equation}
		Use this to argue that $C(t)$ has no signal-to-noise problem, that is $C(t)/\sigma(t)$ (where $\sigma^2(t)$ is variance of $C(t)$) is asymptotically independent of $t$.  
	\item Next compute the correlation function for the $\rho$-meson interpolator, which is the same as the operator above, but with $\gamma_5 \rightarrow \gamma_i$. How does the signal-to-noise ratio behave as a function of $t$ now?
	\item Finally, consider the $\eta$-meson interpolator 
		\begin{equation}
			O_{\eta}(t) = \frac{1}{\sqrt{2}}\sum_{\textbf{x}}\left( \bar{u}(x)\,\gamma_5\,u(x) + \bar{d}(x) \, \gamma_5\, d(x)\right)  
		\end{equation}
		and compute the correlation function in terms of quark propagators. Note that this correlator requires 'disconnected' diagrams, where the quark propagators start and end at the same time. How does the signal-to-noise ratio behave here? Does $\sigma(t)$ fall off with $t$?  
\end{enumerate}