\documentclass[11pt]{latex/exercise}

\setinstitute{Ruhr-University Bochum}
\setsubjectcode{Modern Techniques in Hadron Physics}
\setsubjectname{MTHS24}
\setsemester{Summer 2024}
 
\setlecturers{Vincent Mathieu / Arkaitz Rodas}
\settutors{Daniel Winney, Whytt Smith}
\setsheetnumber{7}
\setexdate{Monday, 22 July 2024}

\begin{document}
\makeheader

\morning

\material{
    \item Partial waves
    \item Analyticity
    \item Unitarity
}{
    \item Gattringer, Lang: Quantum Chromodynamics on the Lattice, \href{https://inspirehep.net/literature/841572}{inSpire}
    \item L{\"u}scher, Computational Strategies in Lattice QCD, \href{arxiv}{https://arxiv.org/pdf/1002.4232}
}


\afternoon

In this exercise session, we will use a Jupyter Notebook to explore various aspects of Lattice QCD.
Follow these steps to set up your environment:

\begin{enumerate}
    \item \textbf{Clone the Notebook Repository}

          First, open a terminal and clone the repository containing the notebook:
          \begin{verbatim}
    git clone https://github.com/username/lattice-qcd-notebook.git
    \end{verbatim}
          Navigate into the repository directory:
          \begin{verbatim}
    cd lattice-qcd-notebook
    \end{verbatim}

    \item \textbf{Install Python in a Conda Environment with Dependencies}

          Make sure you have \texttt{conda} installed. Create a new conda environment and install the required packages:
          \begin{verbatim}
    conda create --name lattice-qcd python=3.9
    conda activate lattice-qcd
    conda install numpy scipy matplotlib jupyterlab
    \end{verbatim}

          Additionally, install any other dependencies listed in the \texttt{requirements.txt} file:
          \begin{verbatim}
    pip install -r requirements.txt
    \end{verbatim}

    \item \textbf{Start Jupyter Lab and Open the Notebook}

          Launch Jupyter Lab from the terminal:
          \begin{verbatim}
    jupyter lab
    \end{verbatim}
          In your web browser, navigate to the Jupyter Lab interface. Open the notebook file \texttt{lattice\_qcd.ipynb}.

    \item \textbf{Jupyter Notebook Cheat Sheet}

          For basic commands and shortcuts in Jupyter Notebook, refer to the [Jupyter Notebook Cheat Sheet](https://jupyter-notebook.readthedocs.io/en/stable/notebook.html).
\end{enumerate}

\importex{PhaseSpaceIII}{Two-body phase space}{}

\end{document}
